\documentclass{jarticle}

\usepackage[ms]{pxchfon}% MSフォントを指定
\usepackage{twocolumn}
%\usepackage[dvi ps]{graphicx} %%画像を読み込む
\usepackage[dvipdfmx]{graphicx} %%画像を読み込む
\usepackage{subfigure}
\usepackage{amsmath}          %%genfrac http://www.biwako.shiga-u.ac.jp/sensei/kumazawa/tex/form006.html
%\usepackage{newtxtext,newtxmath}
\usepackage{ulem}             %%http://biwako.shiga-u.ac.jp/sensei/kumazawa/tex/ulem.html     uline,uuline,uwave,sout,xoutなど
\usepackage{multirow}
\usepackage{chukan2020}       %%最後に読み込むこと!(最後に読み込まないと\textwidthなどの設定が反映されない)

\pagestyle{empty} %ページ番号を入れるときにはコメントアウトする

\begin{document}

\linesparpage{50}

\title{
細径空圧筋を用いた外骨格生物模倣ロボットの開発
}
\etitle{
Development of an Exoskeletal Biomimetic Robot Using Fine Pneumatic Muscles
}
\author{
研究者 濱口 紘生\\
指導教員 中西 大輔
}
\eauthor{
Keywords: McKibben Pneumatic Actuater,Exoskelton,Biomimetic Robot
}

\maketitle

\thispagestyle{empty}  %1ページ目にページ番号を入れるときにはコメントアウトする

%%%%%%%%%%%%%%%%%%%%%%%%%%%%%%%%%%%%%%%%%%%%%%%%%%%%%%%%%%%%%%%%%%%%%%%%%%%%%%%
\section{緒言}

あ

%%%%%%%%%%%%%%%%%%%%%%%%%%%%%%%%%%%%%%%%%%%%%%%%%%%%%%%%%%%%%%%%%%%%%%%%%%%%%%%
\vspace*{-2mm}
\section{犬型ロボット}

本研究では,ROBOTIS社のロボット製作キットBioloidで組み立てた犬型ロボットに強化学習を適用し,起き上がり動作の獲得の実験を行う.
この犬型ロボットにはそれぞれの脚に3個ずつ,全部で12個のモータで構成されている.これら12個のモータを$q_1, \cdots, q_{12}$と表記する.
本研究では,12個のモータすべてを学習対象とするのではなく,$q_1, \cdots, q_6$の6個のモータを学習対象とする.

シミュレーション実験を行う際には,Cyberbotics社のロボットシミュレータソフトウェアWebotsを使用する.
犬型ロボットのシミュレーションモデルには実機ロボットを参考に構築されたモ
デルを使用する.
\if0
\begin{figure}[!b]
  \begin{center}
   \includegraphics[height=33mm,width=70mm]{Fig/Fig1.eps}
   \vspace*{-4mm}
   \caption{図面の例}
   \label{robot}
  \end{center}
\end{figure}
\fi
%%%%%%%%%%%%%%%%%%%%%%%%%%%%%%%%%%%%%%%%%%%%%%%%%%%%%%%%%%%%%%%%%%%%%%%%%%%%%%%
\vspace*{-2mm}
\section{まとめと今後の予定}

卒業研究では,犬型ロボットに強化学習を適用し,特定の初期状態から起き上がり動作を獲得することを実現した.現在,任意の初期状態での起き上がり動作の獲得の方法について検討を行っている.
まず,犬型ロボットの歩行時の転倒パターンのデータを収集した.それらをもとに,主成分分析を用いて,ロボットの転倒パターンの分布を調べ,大まかに3つのパターンに分けられることを確認した.
また,パターンAの犬型ロボットが横転している状態から起き上がる動作を人間がプログラミングすることにより,実際にロボットが起き上がることができることを確認した.

今後の予定として,横転した状態(パターンA)からパターンCに至る動作の獲得を強化学習により実現する.
パターンCから起き上がる動作は卒業研究で獲得済みであるため,最終的に横転した状態からの起き上がりが可能となる.

%%%%%%%%%%%%%%%%%%%%%%%%%%%%%%%%%%%%%%%%%%%%%%%%%%%%%%%%%%%%%%%%%%%%%%%%%%%%%%%
\begin{thebibliography}{99}

\bibitem{Horiuchi2013}
堀内 匡,
NGnetを用いた強化学習によるロボットの行動獲得,
電気学会技術報告「機械学習技術の基礎と応用」,pp.23-27, 2013

\bibitem{Ishikura2014}
石倉裕貴,岸本良一,堀内 匡,
CPGと強化学習を用いた多脚ロボットの行動獲得に関する検討,
電気学会研究会資料,システム研究会ST-13-120, pp.25-28, 2013

\bibitem{Nagami2014}
永海 昂,堀内 匡,
強化学習を用いた四脚ロボットの起き上がり動作の獲得に関する検討,
平成26年電気学会電子・情報・システム部門大会講演論文集,pp.1763-1764,2014

\end{thebibliography}
\end{document}